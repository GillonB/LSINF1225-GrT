\documentclass{scrartcl}
\usepackage[utf8]{inputenc}
\usepackage[T1]{fontenc}      
\usepackage[francais]{babel}
% Layout and figures
%\usepackage[top=2.5cm,bottom=2.5cm,right=2.5cm,left=2.5cm]{geometry}
\usepackage{subfigure}
\usepackage{rotating}
% Units and numbers
\usepackage[squaren, Gray]{SIunits}
\usepackage{sistyle}
\usepackage[autolanguage]{numprint}
% Math
\usepackage{amsmath}
\usepackage{amssymb}
\usepackage{amsthm}
% Links
\usepackage{url}
\usepackage{hyperref}
\hypersetup{
    colorlinks,
    citecolor=black,
    filecolor=black,
    linkcolor=black,
    urlcolor=black
}
% New commands
\newcommand{\annexe}{\part{Annexes}\appendix}
\newcommand{\biblio}[1]{\bibliographystyle{plain}\bibliography{#1}\nocite{*}}

\newcommand{\doctitle}[1]{
	\title{LSINF1225 - Projet BarTender}
	\subtitle{#1}
	\author{\textbf{Groupe T}\\
	\textsc{Gérard} Louis (6317-12-00)\\
	\textsc{Gillon} Bastien (5937-12-00)\\
	\textsc{Jacques} Thibault (0000-13-00)\\
	\textsc{Paris} Antoine (3158-13-00)\\
	\textsc{Ramelot} Sylvain (4763-13-00)}
	\date{\today}

	\begin{document}

	\maketitle
	%\tableofcontents
}

\doctitle{Rapport}

\section{Choix des extensions}
Nous avons décidé de choisir des extensions pour faciliter
le choix de boissons par les clients.

Dès lors, deux idées principales sont venues à nous. 
Premièrement, un accès facile aux goûts du client par 
le choix de différentes catégories interactives.

Ainsi, nous imaginons une ``Recherche avancées'' via une carte 
découpée en catégories et sous-catégories permettant aux clients de
réduire le choix selon des critères qu'ils détermineraient. 

Nous avons également pensé à un système d'avis pour chaque boisson. 
Un client peut donner son avis sur une boisson en donnant une note
sur 5 et en laissant, s'il le souhaite, un commentaire. Cette
fonctionnalité permettra, par exemple, de guider un client hésitant
entre deux boissons.

Ainsi, nos extensions sont principalement basées sur une utilisation
optimale de la carte pour le client.

Nous pouvons ajouter à cela, un choix de conception concernant l'ORM et plus particulièrement, 
l'entité "Client" et ses valeurs :"login_name" et "password. Il n'y a pas de contrainte d'obligation sur le Client 
car nous avons choisi de faire un bouton carte toujours accessible, même pour une personne non "logger".
L'identifiant et le mot de passe du client sera surtout utile pour les habitués lesquel auront des fonctionnalités supplémentaires
comme noter les boissons et, sans obligation, pourront laisser un commentaire.

\end{document}
