\documentclass{scrartcl}
\usepackage[utf8]{inputenc}
\usepackage[T1]{fontenc}      
\usepackage[francais]{babel}
% Layout and figures
%\usepackage[top=2.5cm,bottom=2.5cm,right=2.5cm,left=2.5cm]{geometry}
\usepackage{subfigure}
\usepackage{rotating}
% Units and numbers
\usepackage[squaren, Gray]{SIunits}
\usepackage{sistyle}
\usepackage[autolanguage]{numprint}
% Math
\usepackage{amsmath}
\usepackage{amssymb}
\usepackage{amsthm}
% Links
\usepackage{url}
\usepackage{hyperref}
\hypersetup{
    colorlinks,
    citecolor=black,
    filecolor=black,
    linkcolor=black,
    urlcolor=black
}
% New commands
\newcommand{\annexe}{\part{Annexes}\appendix}
\newcommand{\biblio}[1]{\bibliographystyle{plain}\bibliography{#1}\nocite{*}}

\newcommand{\doctitle}[1]{
	\title{LSINF1225 - Projet BarTender}
	\subtitle{#1}
	\author{\textbf{Groupe T}\\
	\textsc{Gérard} Louis (6317-12-00)\\
	\textsc{Gillon} Bastien (5937-12-00)\\
	\textsc{Jacques} Thibault (0000-13-00)\\
	\textsc{Paris} Antoine (3158-13-00)\\
	\textsc{Ramelot} Sylvain (4763-13-00)}
	\date{\today}

	\begin{document}

	\maketitle
	%\tableofcontents
}

\doctitle{Entités et faits élementaire}
On distingue 7 entités dont une est due à une
extension choisie. Il y a des relations unaires
(lien entre entité et valeur), des relations binaires
(lien entre deux entités) et une seule relation ternaire.

\paragraph{Serveur (identifiant) anparis}
\begin{itemize}
	\item ...\textbf{s'appelle} ``Antoine'' ;
	\item ...\textbf{a pour mot de passe} ``ninja007'' ;
	\item ...\textbf{prend} Commande (numéro) 4523.
\end{itemize}

\paragraph{Client (identifiant) kmens}
\begin{itemize}
	\item ...\textbf{a pour mot de passe} ``ORM\_is\_life'' ;
	\item ...\textbf{est assis à} Table (numéro) 4.
\end{itemize}

\paragraph{Boisson (référence) JUP33}
\begin{itemize}		
	\item ...\textbf{a pour nom} ``Jupiler'' ;
	\item ...\textbf{a pour description} ``La Jupiler est une bière belge blonde
	de fermentation basse de type pils'' ;
	\item ...\textbf{coûte} 2.50 euros ;
	\item ...\textbf{contient} 33 cl ;
	\item ...\textbf{appartient à la catégorie} ``Bières'' ;
	\item ...\textbf{appartient à sous-catégorie} ``blondes'' ;
	\item ...\textbf{est représentée par l'image} ``icon/jup.jpg'' ;
	\item ...\textbf{possède un stock actuel de} 50 ;
	\item ...\textbf{possède un stock minimum de}  24 ;
	\item ...\textbf{possède un stock maximum de} 200.
\end{itemize}

\paragraph{Commande (numéro) 4523}
\begin{itemize}
	\item ...\textbf{est passée par} Table (numéro) 4 ;
	\item ...\textbf{est prise par} Serveur (identifiant) anparis ;
	\item ...\textbf{est liée à} Addition (numéro) 623 ;
	\item ...\textbf{contient} Boisson (référence) JUP33 \textbf{en quantité} 3 ;		
	\item ...\textbf{a été passée à} ``05/03/2015 à 20:57''.
\end{itemize}

\paragraph{Addition (numéro) 623}
\begin{itemize}
	\item ...\textbf{a été prise à} ``05/03/2015 à 21:12'' ;
	\item ...\textbf{contient} Commande (numéro) 4523.
\end{itemize}

\paragraph{Table (numéro) 4}
\begin{itemize}
	\item ...\textbf{accueille} Client (identifiant) kmens ;
	\item ...\textbf{passe} Commande (numéro) 4523 ;
\end{itemize}

\paragraph{Avis (numéro) 714}
\begin{itemize}
	\item ...\textbf{est associé à} Boisson (référence) JUP33;
	\item ...\textbf{contient la cote} 5 ;
	\item ...\textbf{contient le commentaire} ``Excellente bière !'' ;
	\item ...\textbf{est donné par} Client (identifiant) kmens.
\end{itemize}

\end{document}